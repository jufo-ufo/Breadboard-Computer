\documentclass{scrartcl}

\usepackage{graphicx}
\usepackage[siunitx, RPvoltages, european]{circuitikz}

\usepackage[ngerman]{babel}
\usepackage[utf8]{inputenc}
\usepackage{amsmath}
\usepackage{amssymb}
\usepackage[T1]{fontenc}
\usepackage{xcolor}
\usepackage{tikz}
\usepackage[breaklinks=true]{hyperref}
\usepackage[utf8]{inputenc}
\usepackage[babel,german=quotes]{csquotes}
\usepackage[style=numeric, backend=biber]{biblatex}
\usepackage{listings}
\usepackage{bytefield}
\usepackage{longtable}

\addbibresource{ref.bib}

\graphicspath{ {../images} }

\KOMAoptions{parskip=full}


\begin{document}

    \title{Platinencomputer - Architektur}
    \author{Alexander Wersching und Simon Walter}
    \date{2021}
    \maketitle

    \newpage
    \tableofcontents
    \newpage

    \section{Einführung}

    \section{Grundlegende Design-Prinzipien}
    
    \subsection{Laufzeitverzögerung}
    
    \subsection{Synchrone Speicherung von Daten}

    Im Computer finden sich jedoch nicht nur simple combintatorische Logik-Schaltungen wie Addition oder XOR. Viel mehr speicher und veränder der Computer Daten. Jede (oder ein großer Teil) der Schalungen lassen sich auf einen Aufbau wie ... zusammenfassen.

    Die Frage ist nur noch wie das Beschreiben des Register funktioniniert. Wir nutzen dafür ein Clock (einen periodisch wieder kehrenden Pulsschlag). Wobei wir das Beschreiben auf dieses Clock-Schlag abstimmen. Die einfachste Methode wäre es das Register zubeschreiben, wenn das Clock-Singal eine logische $1$ darstellt. So ein Aufbau ist in ... gezeigt.

    Das Problem mit ... ist leider nur das es zu einer Oszillation kommt. Die kombinatiorische Schaltung, indem Fall ein bit-wiese NOT, hat, wie alle kombinatiorischen Schaltunen ein Laufzeitverzögerung. Das wenn sich, nach abgelaufender Laufzeitverzögerung, das Singal am Ausgang der kombinatorschen Schaltung ändert, wird diese Änderung direkt in das Register, geschrieben, bei welchem sicher der Ausgang veränder. Diese ändert wieder den Eingangswert der kombinatiorischen Schaltung, welcher wieder nach abgelaufender Laufzeitverzögerung den Ausgang der kombinatiorische Schaltung änderet, usw. 
    
    Dabei ergibt sich für die Oszilation ein simple Regel. Nehmen wir an wir nutzen ein Singal wie in ... gezeigt. Sogilt für die Anzahl der Änderungen im Register

    \section{Komponenten des Computers}

    \subsubsection{Activation Module}

    Das Activation Module (kurz AM) ist eine wichtige Komponenten welche zur Steuerung der Datenfluss Richtung dient. Der Schmatische Aufbau ist in \autoref{fig:AM_schem} gezeigt. Das AM wird an der ctrl\_line entweder in die Write- oder Read-Kontrolllinie eines Buses eingesteckt.
    
    Auf diesem Module sind Dip-Schalter befestigt, welche an den ctrl\_seq Eingang angeschlossen sind, mit welchen sich einstellen lässt auf welches signal das AM hören soll. Wenn nun die Eingestellt Kontrollsequenz auf den Kontrolllinien anliegt, gibt das AM einen Signal an die Angeschlossene Komponente ab. 
    
    Zudem lässt sich über ein weiteren Dip-Schalter, welcher an Mode angeschlossen ist, einstellen ob das Module im Read oder Write Modus arbeiten soll. Im Read Modus gibt das Module auf seinem Signal\_Pin\_1 und Signal\_Pin\_2 solange eine Read Signal aus, bist die Sequenz nicht mehr auf dem Bus anliegt. Bei Write, wird ein Kurzer Puls auf Signal\_Pin\_1 abgegeben wenn das AM eine steigende Kannte des Clock Signals regestiert und die richtige Kontrolsequenz auf dem Kontrolllinen anliegt. Auf Singal\_Pin\_2 wird eine kurzer Puls abgegeben wenn das AM eine Fallende Kannte regestiert und die richtige Kontrollsequenz auf den Kontroll linien Anliegt.

    Wichtig zu wissen ist, das das AM eine rein kobinatorsche Schaltung ist. Daher hat es auch eine feste $t_{PD}$. Dieser kann wie folgt rechent werden: Wie in ... gezeigt, ist der Durschnittlich gemessen $t_{PD} = \text{some value}$. Wobei das gemessen maximum und minimum sind.
    
    \begin{figure}
        \centering
        \resizebox{14cm}{!}{% Important: If latex complains about unicode characters,
% please use "\usepackage[utf8x]{inputenc}" in your preamble
% You can change the size of the picture by putting it into the construct:
% 1) \resizebox{10cm}{!}{"below picture"} to scale horizontally to 10 cm
% 2) \resizebox{!}{15cm}{"below picture"} to scale vertically to 15 cm
% 3) \resizebox{10cm}{15cm}{"below picture"} a combination of above two
% It is not recomended to use the scale option of the tikzpicture environment.
\begin{tikzpicture}[x=1pt,y=-1pt,line cap=rect]
\def\logisimfontA#1{\fontfamily{cmr}{#1}} % Replaced by logisim, original font was "SansSerif"
\definecolor{custcol_0_0_0}{RGB}{0, 0, 0}
\definecolor{custcol_ff_ff_ff}{RGB}{255, 255, 255}
\draw [line width=3.0pt, custcol_0_0_0 ]  (248.0,95.0) -- (348.0,95.0) ;
\draw [line width=3.0pt, custcol_0_0_0 ]  (248.0,25.0) -- (348.0,25.0) ;
\draw [line width=3.0pt, custcol_0_0_0 ]  (88.0,125.0) -- (118.0,125.0) ;
\draw [line width=3.0pt, custcol_0_0_0 ]  (378.0,105.0) -- (398.0,105.0) ;
\draw [line width=3.0pt, custcol_0_0_0 ]  (378.0,35.0) -- (398.0,35.0) ;
\draw [line width=4.0pt, custcol_0_0_0 ]  (88.0,185.0) -- (108.0,185.0) ;
\draw [line width=4.0pt, custcol_0_0_0 ]  (88.0,205.0) -- (108.0,205.0) ;
\draw [line width=3.0pt, custcol_0_0_0 ]  (338.0,115.0) -- (348.0,115.0) ;
\draw [line width=3.0pt, custcol_0_0_0 ]  (308.0,195.0) -- (338.0,195.0) -- (338.0,115.0) -- (338.0,45.0) -- (348.0,45.0) ;
\draw [line width=4.0pt, custcol_0_0_0 ]  (148.0,195.0) -- (158.0,195.0) ;
\fill [line width=3.0pt, custcol_0_0_0]  (338.0,115.0) ellipse (5.0 and 5.0 );
\fill [line width=3.0pt, custcol_0_0_0]  (178.0,105.0) ellipse (5.0 and 5.0 );
\draw [line width=2.0pt, custcol_0_0_0 ]  (70.0,77.0) -- (87.0,77.0) ;
\draw [line width=2.0pt, custcol_0_0_0 ]  (88.0,77.0) -- (88.0,94.0) ;
\draw [line width=2.0pt, custcol_0_0_0 ]  (88.0,95.0) -- (71.0,95.0) ;
\draw [line width=2.0pt, custcol_0_0_0 ]  (70.0,95.0) -- (70.0,78.0) ;
\logisimfontA{\fontsize{12pt}{12pt}\selectfont\node[inner sep=0, outer sep=0, custcol_0_0_0, anchor=base west] at  (72.0,92.0)  {x1};}
\logisimfontA{\fontsize{16pt}{16pt}\fontseries{bx}\selectfont\node[inner sep=0, outer sep=0, custcol_0_0_0, anchor=base west] at  (13.0,93.0)  {CLK\_F};}
\fill [line width=2.0pt, custcol_0_0_0]  (88.0,85.0) ellipse (2.0 and 2.0 );
\draw [line width=2.0pt, custcol_0_0_0 ]  (70.0,117.0) -- (87.0,117.0) ;
\draw [line width=2.0pt, custcol_0_0_0 ]  (88.0,117.0) -- (88.0,134.0) ;
\draw [line width=2.0pt, custcol_0_0_0 ]  (88.0,135.0) -- (71.0,135.0) ;
\draw [line width=2.0pt, custcol_0_0_0 ]  (70.0,135.0) -- (70.0,118.0) ;
\logisimfontA{\fontsize{12pt}{12pt}\selectfont\node[inner sep=0, outer sep=0, custcol_0_0_0, anchor=base west] at  (72.0,132.0)  {x1};}
\logisimfontA{\fontsize{16pt}{16pt}\fontseries{bx}\selectfont\node[inner sep=0, outer sep=0, custcol_0_0_0, anchor=base west] at  (24.0,133.0)  {Mode};}
\fill [line width=2.0pt, custcol_0_0_0]  (88.0,125.0) ellipse (2.0 and 2.0 );
\draw [line width=2.0pt, custcol_0_0_0 ]  (70.0,177.0) -- (87.0,177.0) ;
\draw [line width=2.0pt, custcol_0_0_0 ]  (88.0,177.0) -- (88.0,194.0) ;
\draw [line width=2.0pt, custcol_0_0_0 ]  (88.0,195.0) -- (71.0,195.0) ;
\draw [line width=2.0pt, custcol_0_0_0 ]  (70.0,195.0) -- (70.0,178.0) ;
\logisimfontA{\fontsize{12pt}{12pt}\selectfont\node[inner sep=0, outer sep=0, custcol_0_0_0, anchor=base west] at  (72.0,192.0)  {x8};}
\logisimfontA{\fontsize{16pt}{16pt}\fontseries{bx}\selectfont\node[inner sep=0, outer sep=0, custcol_0_0_0, anchor=base west] at  (6.0,193.0)  {ctrl\_line};}
\fill [line width=2.0pt, custcol_0_0_0]  (88.0,185.0) ellipse (2.0 and 2.0 );
\draw [line width=2.0pt, custcol_0_0_0 ]  (70.0,197.0) -- (87.0,197.0) ;
\draw [line width=2.0pt, custcol_0_0_0 ]  (88.0,197.0) -- (88.0,214.0) ;
\draw [line width=2.0pt, custcol_0_0_0 ]  (88.0,215.0) -- (71.0,215.0) ;
\draw [line width=2.0pt, custcol_0_0_0 ]  (70.0,215.0) -- (70.0,198.0) ;
\logisimfontA{\fontsize{12pt}{12pt}\selectfont\node[inner sep=0, outer sep=0, custcol_0_0_0, anchor=base west] at  (72.0,212.0)  {x8};}
\logisimfontA{\fontsize{16pt}{16pt}\fontseries{bx}\selectfont\node[inner sep=0, outer sep=0, custcol_0_0_0, anchor=base west] at  (5.0,213.0)  {ctrl\_seq};}
\fill [line width=2.0pt, custcol_0_0_0]  (88.0,205.0) ellipse (2.0 and 2.0 );
\draw [line width=2.0pt, custcol_0_0_0 ]  (70.0,7.0) -- (87.0,7.0) ;
\draw [line width=2.0pt, custcol_0_0_0 ]  (88.0,7.0) -- (88.0,24.0) ;
\draw [line width=2.0pt, custcol_0_0_0 ]  (88.0,25.0) -- (71.0,25.0) ;
\draw [line width=2.0pt, custcol_0_0_0 ]  (70.0,25.0) -- (70.0,8.0) ;
\logisimfontA{\fontsize{12pt}{12pt}\selectfont\node[inner sep=0, outer sep=0, custcol_0_0_0, anchor=base west] at  (72.0,22.0)  {x1};}
\logisimfontA{\fontsize{16pt}{16pt}\fontseries{bx}\selectfont\node[inner sep=0, outer sep=0, custcol_0_0_0, anchor=base west] at  (11.0,23.0)  {CLK\_R};}
\fill [line width=2.0pt, custcol_0_0_0]  (88.0,15.0) ellipse (2.0 and 2.0 );
\draw [line width=3.0pt, custcol_0_0_0 ]  (228.0,155.0) -- (178.0,155.0) -- (164.0,155.0) ;
\draw [line width=3.0pt, custcol_0_0_0 ]  (228.0,235.0) -- (188.0,235.0) -- (188.0,225.0) -- (178.0,225.0) -- (164.0,225.0) ;
\draw [line width=5.0pt, custcol_0_0_0 ]  (159.0,195.0) -- (164.0,195.0) ;
\draw [line width=5.0pt, custcol_0_0_0 ]  (164.0,156.0) -- (164.0,224.0) ;
\logisimfontA{\fontsize{7pt}{7pt}\selectfont\node[inner sep=0, outer sep=0, custcol_0_0_0, anchor=base west] at  (167.0,152.0)  {0};}
\logisimfontA{\fontsize{7pt}{7pt}\selectfont\node[inner sep=0, outer sep=0, custcol_0_0_0, anchor=base west] at  (167.0,162.0)  {1};}
\logisimfontA{\fontsize{7pt}{7pt}\selectfont\node[inner sep=0, outer sep=0, custcol_0_0_0, anchor=base west] at  (167.0,172.0)  {2};}
\logisimfontA{\fontsize{7pt}{7pt}\selectfont\node[inner sep=0, outer sep=0, custcol_0_0_0, anchor=base west] at  (167.0,182.0)  {3};}
\logisimfontA{\fontsize{7pt}{7pt}\selectfont\node[inner sep=0, outer sep=0, custcol_0_0_0, anchor=base west] at  (167.0,192.0)  {4};}
\logisimfontA{\fontsize{7pt}{7pt}\selectfont\node[inner sep=0, outer sep=0, custcol_0_0_0, anchor=base west] at  (167.0,202.0)  {5};}
\logisimfontA{\fontsize{7pt}{7pt}\selectfont\node[inner sep=0, outer sep=0, custcol_0_0_0, anchor=base west] at  (167.0,212.0)  {6};}
\logisimfontA{\fontsize{7pt}{7pt}\selectfont\node[inner sep=0, outer sep=0, custcol_0_0_0, anchor=base west] at  (167.0,222.0)  {7};}
\fill [line width=5.0pt, custcol_0_0_0]  (158.0,195.0) ellipse (2.0 and 2.0 );
\fill [line width=5.0pt, custcol_0_0_0]  (178.0,155.0) ellipse (2.0 and 2.0 );
\fill [line width=5.0pt, custcol_0_0_0]  (178.0,165.0) ellipse (2.0 and 2.0 );
\fill [line width=5.0pt, custcol_0_0_0]  (178.0,175.0) ellipse (2.0 and 2.0 );
\fill [line width=5.0pt, custcol_0_0_0]  (178.0,185.0) ellipse (2.0 and 2.0 );
\fill [line width=5.0pt, custcol_0_0_0]  (178.0,195.0) ellipse (2.0 and 2.0 );
\fill [line width=5.0pt, custcol_0_0_0]  (178.0,205.0) ellipse (2.0 and 2.0 );
\fill [line width=5.0pt, custcol_0_0_0]  (178.0,215.0) ellipse (2.0 and 2.0 );
\fill [line width=5.0pt, custcol_0_0_0]  (178.0,225.0) ellipse (2.0 and 2.0 );
\draw [line width=2.0pt, custcol_0_0_0]  (409.0,106.0) ellipse (9.0 and 9.0 );
\logisimfontA{\fontsize{12pt}{12pt}\selectfont\node[inner sep=0, outer sep=0, custcol_0_0_0, anchor=base west] at  (402.0,112.0)  {x1};}
\logisimfontA{\fontsize{16pt}{16pt}\fontseries{bx}\selectfont\node[inner sep=0, outer sep=0, custcol_0_0_0, anchor=base west] at  (420.0,113.0)  {Signal\_Out\_2};}
\fill [line width=2.0pt, custcol_0_0_0]  (398.0,105.0) ellipse (2.0 and 2.0 );
\draw [line width=2.0pt, custcol_0_0_0]  (409.0,36.0) ellipse (9.0 and 9.0 );
\logisimfontA{\fontsize{12pt}{12pt}\selectfont\node[inner sep=0, outer sep=0, custcol_0_0_0, anchor=base west] at  (402.0,42.0)  {x1};}
\logisimfontA{\fontsize{16pt}{16pt}\fontseries{bx}\selectfont\node[inner sep=0, outer sep=0, custcol_0_0_0, anchor=base west] at  (420.0,43.0)  {Signal\_Out\_1};}
\fill [line width=2.0pt, custcol_0_0_0]  (398.0,35.0) ellipse (2.0 and 2.0 );
\draw [line width=2.0pt, custcol_0_0_0 ]  (138.0,125.0) -- (119.0,118.0) -- (119.0,132.0) -- cycle;
\draw [line width=2.0pt, custcol_0_0_0]  (143.0,125.0) ellipse (4.5 and 4.5 );
\fill [line width=2.0pt, custcol_0_0_0]  (148.0,125.0) ellipse (2.0 and 2.0 );
\fill [line width=2.0pt, custcol_0_0_0]  (118.0,125.0) ellipse (2.0 and 2.0 );
\draw [line width=3.0pt, custcol_0_0_0 ]  (108.0,185.0) -- (110.0,185.0) ;
\draw [line width=3.0pt, custcol_0_0_0 ]  (108.0,205.0) -- (110.0,205.0) ;
\draw [line width=2.0pt, custcol_0_0_0 ]  (148.0,195.0) .. controls  (138.0,180.0)  ..  (118.0,180.0) .. controls  (126.0,195.0)  ..  (118.0,210.0) .. controls  (138.0,210.0)  ..  (148.0,195.0) -- cycle ;
\draw [line width=2.0pt, custcol_0_0_0 ]  (108.0,180.0) .. controls  (116.0,195.0)  ..  (108.0,210.0) ;
\draw [line width=3.0pt, custcol_0_0_0 ]  (88.0,85.0) -- (218.0,85.0) -- (220.0,85.0) ;
\draw [line width=3.0pt, custcol_0_0_0 ]  (178.0,105.0) -- (218.0,105.0) -- (220.0,105.0) ;
\draw [line width=2.0pt, custcol_0_0_0 ]  (248.0,95.0) .. controls  (238.0,80.0)  ..  (218.0,80.0) .. controls  (226.0,95.0)  ..  (218.0,110.0) .. controls  (238.0,110.0)  ..  (248.0,95.0) -- cycle ;
\draw [line width=3.0pt, custcol_0_0_0 ]  (88.0,15.0) -- (218.0,15.0) -- (220.0,15.0) ;
\draw [line width=3.0pt, custcol_0_0_0 ]  (148.0,125.0) -- (178.0,125.0) -- (178.0,105.0) -- (178.0,35.0) -- (218.0,35.0) -- (220.0,35.0) ;
\draw [line width=2.0pt, custcol_0_0_0 ]  (248.0,25.0) .. controls  (238.0,10.0)  ..  (218.0,10.0) .. controls  (226.0,25.0)  ..  (218.0,40.0) .. controls  (238.0,40.0)  ..  (248.0,25.0) -- cycle ;
\draw [line width=3.0pt, custcol_0_0_0 ]  (164.0,165.0) -- (178.0,165.0) -- (228.0,165.0) -- (230.0,165.0) ;
\draw [line width=3.0pt, custcol_0_0_0 ]  (164.0,175.0) -- (178.0,175.0) -- (228.0,175.0) -- (234.0,175.0) ;
\draw [line width=3.0pt, custcol_0_0_0 ]  (164.0,185.0) -- (178.0,185.0) -- (228.0,185.0) -- (237.0,185.0) ;
\draw [line width=3.0pt, custcol_0_0_0 ]  (164.0,195.0) -- (178.0,195.0) -- (218.0,195.0) -- (218.0,205.0) -- (228.0,205.0) -- (237.0,205.0) ;
\draw [line width=3.0pt, custcol_0_0_0 ]  (164.0,205.0) -- (178.0,205.0) -- (208.0,205.0) -- (208.0,215.0) -- (228.0,215.0) -- (234.0,215.0) ;
\draw [line width=3.0pt, custcol_0_0_0 ]  (164.0,215.0) -- (178.0,215.0) -- (198.0,215.0) -- (198.0,225.0) -- (228.0,225.0) -- (230.0,225.0) ;
\draw [line width=2.0pt, custcol_0_0_0 ]  (298.0,195.0) .. controls  (273.0,160.0)  ..  (228.0,160.0) .. controls  (248.0,195.0)  ..  (228.0,230.0) .. controls  (273.0,230.0)  ..  (298.0,195.0) -- cycle ;
\draw [line width=2.0pt, custcol_0_0_0 ]  (228.0,150.0) .. controls  (230.0,155.0)  ..  (228.0,160.0) .. controls  (248.0,195.0)  ..  (228.0,230.0) .. controls  (230.0,235.0)  ..  (228.0,240.0) ;
\draw [line width=2.0pt, custcol_0_0_0]  (303.0,195.0) ellipse (4.5 and 4.5 );
\draw [line width=2.0pt, custcol_0_0_0] (363.0,120.0) arc (90.0:-90.0:15.0 and 15.0 );
\draw [line width=2.0pt, custcol_0_0_0 ]  (363.0,90.0) -- (349.0,90.0) -- (349.0,120.0) -- (363.0,120.0) ;
\draw [line width=2.0pt, custcol_0_0_0] (363.0,50.0) arc (90.0:-90.0:15.0 and 15.0 );
\draw [line width=2.0pt, custcol_0_0_0 ]  (363.0,20.0) -- (349.0,20.0) -- (349.0,50.0) -- (363.0,50.0) ;
\end{tikzpicture}

}
        \caption{Schematik des Activation Module\label{fig:AM_schem}}
    \end{figure}


    \section{Architektur des Computers}


\end{document}